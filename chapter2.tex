\section{destroying entry to dummy}

\qquad 2005/6/17 Yotsuya lergue \#17
\begin{quote}
\crdima{W/None}{%
  \begin{minipage}[t]{\br}
     3NT\\OL;\s Q
  \end{minipage}}%
  {\hand{A7}{KQ1086}{K6}{8762}}%
  {\hand{KJ85}{J7}{Q}{KQ10953}}
  {\hand{9642}{A9532}{J853}{-}}%
  {\hand{Q103}{4}{A109742}{AJ4}}%
\end{quote}
\begin{quote}
\begin{bidding}
1\c  \> 1\h \> P \> 2\d\\
P \> 2\h \>P \> 2NT \\
P \> 3NT \> a.p.
\end{bidding}
 \end{quote}

{\bf Remarks}\\

Opening leas is \c Q. East discards \d 2. South ducks.
What is your lead for second trick?

Do not give up to imagine south's hand.
Let's have a habit to estimate logically.

In order to rot dummy's heart, west led \s K which
consumed an entry to dummy. South cashed \d K and \d A.
He gave up to establish diamond due to bad break.So he led
\h 4.West plyed \h 7 (\d J might show the doubleton.)
Dummy \h K. Professor sitting at east won with ace at once
and returned \s 9. Finnally contract went to down one.

Let's try active defense with rich imagination,

Nest deal is another example to eliminate an entyr to
dummy. It requires deep reading.

\vspace{0.5cm}

%--------28---------------------------------
\qquad 2007/9/15 Princess Takamatu Cop \#18
\begin{quote}
\crdima{S/None}{%
  \begin{minipage}[t]{\br}
     4\h\\OL:\s Q
  \end{minipage}}%
  {\hand{94}{A1052}{Q97543}{7}}%
  {\hand{QJ1085}{Q4}{A2}{K1096}}
  {\hand{63}{87}{K108}{AQJ532}}%
  {\hand{AK72}{KJ963}{J6}{84}}%
\end{quote}
\begin{quote}
\begin{bidding}
- \> -  \> - \> 1\h  \\
1\s  \> 2\h \> P \> 4\h\\
a.p.
\end{bidding}
 \end{quote}

{\bf Remarks}\\

South won the opening lead \s Q with ace. After he
drew trumps twice, he started to establish dummy's
diamond. East won with \d 10. East cashed his \c A.
West played six. He was afraid of  
 \c 10 which became miss if south
has queen and jack. So east returd a spade, This was a
crital error.

South led \d J . West won with ace.
All west had to do is to lead a black suit,
allowing north's ruff. Diamond is rffed by south.
North has a trump entry to run diamond.

In spite of critical error, east's club could consume
dummy's trump. South is worry about dummy has not 
an entry to run diamond.

West might be acused for his no signal.


Next deal shows a blow in order to avoid 
being squeezed.\\
\qquad 2007/9/15 Takamatu mem cup \#19\\
\begin{quote}
\crdima{S/E-W}{%
  \begin{minipage}[t]{\br}
     4\h\\OL: \s Q
  \end{minipage}}%
  {\hand{AJ9543}{963}{2}{1074}}%
  {\hand{KQ106}{QJ76}{J65}{52}}
  {\hand{72}{K4}{KQ10873}{J96}}%
  {\hand{8}{A1082}{A94}{AKQ83}}%
\end{quote}
\begin{quote}
\begin{bidding}
- \> -  \> 2\d \> 3\c  \\
P \> 3\s \> P \> 3NT\\
a.p.
\end{bidding}
\end{quote}


{\bf Remarks}\\


Along the routine bidding, contract reached to 3NT.
Opening leads is \d 5 and east'queen wins. As south denied 5 heart
cards, east shifts to \h K. South ducks and east continues , South
ducks once more and west wins with \h J. West leads \d J , south
allows \d J to win.

If west continuse diamond involuntarily, outrageous trouble will
happen. South has eight winners(1+1+1+5) and has rectified in 4 loosers
and an entry to dummy remains. This circumstace succeeds to squeeze east.

So west's \s K ,in stead of third diamond ,is a blow in order to avoid 
being squeezed. Contract went to one down.

In postmotem south found an end play by taking \d J  and by 
running  all clubs. Throw-in to west by spade forces west to lead a
heart in south's tenese.

Another postmotem found a spade lead before \d J.

There was a coutertrick over a countetrick.

Moreover discussion about second trick (\h K ) was done,
concluding 
west's diamond return was correct. 
%-------------------31--------------------
\vspace{0.5cm}

Next deal appeared in APBF\footnote{
Asia Pacific Bridge Federation
}\\
\index{APBF}
\qquad 2000/9/2 APBF Fukuoka \#39
\begin{quote}
\crdima{N/None}{%
  \begin{minipage}[t]{\br}
     3NT\\OL:\s 10
  \end{minipage}}%
  {\hand{AQ9}{9Q732}{J762}{83}}%
  {\hand{104}{10865}{A43}{J654}}
  {\hand{7654}{QJ4}{KQ10}{Q109}}%
  {\hand{KJ82}{AK}{985}{AK72}}%
\end{quote}
\begin{quote}
\begin{bidding}
- \> -  \> - \> 1\c  \\
P \> 1\h \> P \> 2NT\\
P \> 3NT \> a.p.
\end{bidding}
\end{quote}
{\bf Remarks}\\

Opening lead is \s 10. South wins in hand. \d 8 is taken by
east's \d 10.

East pondered for a long time. He thought  \h trick was 
necessary. He led \h Q ,  South led diamond again .
Finnaly south established thirteenth diamond.

East should realize why south played weak diamond, He wanted 
thirteenth diamond! To break his plan spade return was necessay.
If east returned a sapde, contract failed automatically .

Players in APBF final round would notice such a plan.

%---------------32 page---------------------
\section{no entry to my hand}

\qquad 2006/10/21 Yokohama BC cup \#24
\begin{quote}
\crdima{E/None}{%
  \begin{minipage}[t]{\br}
     3NT\\OL:\c K
  \end{minipage}}%
  {\hand{Q10}{964}{86}{KQJ832}}%
  {\hand{63}{AQ83}{KQ97}{954}}
  {\hand{KJ9875}{1052}{103}{76}}%
  {\hand{A42}{KJ7}{AJ542}{A10}}%
\end{quote}
\begin{quote}
\begin{bidding}
- \> -  \> 2\s \> 2NT \\
P \> 3NT \>a.p.
\end{bidding}
\end{quote}
{\bf Remarks}\\

Opening lead is \s 6. East wins with jack. South ducked against 
\s K .It is quite clear that his spades are useless. He glanced the 
dummy, Since diamond is shorter than heart, he decided to return
\d 10, South covered with jack and west won with queen. West returned \d K.
South can collect 8 tricks only going to down one.

South regreted his ducking at secound trick. Since east's two spade bid 
implied 6 cards south can take it. 
By running six clubs, west might fall in trouble.
At the last club , west in stripped squeeze. By throwing with diamond, west was
oblidged to lead a heart from his hand,

In order to avoid the squeeze, east has a nice return \d 10 at trick two.
This circumstace gives west an exit card in spade.

%-----------------------33 page----------------
\vspace{0.5cm}
Following is another example of no entry to myself.\\


\qquad 2006/8/30 Yokohama sectional \#17
\begin{quote}
\crdima{E/None}{%
  \begin{minipage}[t]{\br}
     3NT:\\OL:\s 6
  \end{minipage}}%
  {\hand{J}{AK83}{KQJ195}{973}}%
  {\hand{742}{764}{872}{AKQ5}}
  {\hand{AK10986}{J83}{94}{J8}}%
  {\hand{Q54}{Q102}{A63}{10642}}%
\end{quote}
\begin{quote}
\begin{bidding}
- \> -  \> 2\s \> P  \\
3\s \> Dbl \> P \> 3NT\\
a.p.
\end{bidding}
\end{quote}
{\bf Remarks}\\

Bidding was same in both tables. At a table, east ducked 
opening sapde lead. South was happy when he cashed 9 tricks 
at once, East had better to win  \s K and attacks club suit.

At another table, west led \c K. Then cased \c Q falling \c J from west.
Alas west led spade, He was afraid of giving south a club trick 
if he cashed \c A.

There are many ways for succeeding defense.
West cashes 3 top clubs. West leads spade at trick 2. 


Following deal shows to keep entry neglecting
"third high".
%------------------34---------------------

\vspace{0.5cm}
\qquad 2006/10/30 Yokohama sectional \#16
\begin{quote}
\crdima{N/N-S}{%
  \begin{minipage}[t]{\br}
     4\h:\\OL:\s 8
  \end{minipage}}%
  {\hand{Q42}{J985}{AJ}{KJ92}}%
  {\hand{87}{Q72}{K94}{AQ1043}}
  {\hand{A963}{K}{106532}{865}}%
  {\hand{KJ105}{A10643}{JQ87}{7}}%
\end{quote}
\begin{quote}
\begin{bidding}
- \> 1\c  \> P  \> 1\h  \\
P \> 2\h \> P  \> 3\h\\
P \> 4\h \> a.p.
\end{bidding}
\end{quote}

{\bf Remarks}\\

Opening lead is \s 8. West wondered to play ace.
If west is sigleton , ace is correct. If douleton . ace must
be reserved to trasfer control.

In this case , as south's bid denies to have 5 cards spade.
west played \s 9.

South led \h  A then  \h 3. West won with \h Q . East discarded \s 3;
West made a mistake believing the saying "Third hand highest", West
thought east has not \s A. How do you think about west's excuse ?




%-----------------35 page---------------------
\section{find secure entry to partner}

\qquad 2004/2/13 Sibutani Cup \#20
\begin{quote}
\crdima{N/None}{%
  \begin{minipage}[t]{\br}
     Play:\\demo
  \end{minipage}}%
  {\hand{9}{K1053}{852}{K7542}}%
  {\hand{Q8763}{A64}{K843}{K7542}}
  {\hand{KJ64}{8}{AQ1097}{Q86}}%
  {\hand{A102}{QJ972}{J}{AJ109}}%
\end{quote}
\begin{quote}
\begin{bidding}
- \> - \> 1\d   \> 1\h  \\
1\s \> 2\h \> 2\s  \> 3\c\\
3\s \> 4\c \> P \> P \\
4\d \> 4\h \> P \> P \\
4\s \> 5\c \> Dbl \> P \\
P \> 5\h \> Dbl \> a.p.
\end{bidding}
\end{quote}
{\bf Remarks}\\

Opening lead is singleton \c 3. Dummy wins with king and 
leads \h 3 toward \h Q. West won with \h A. West made a
mistake by returning spade. South made doubled contract.

West did not make an effort to know an entry to his parter.
West should take \h A in next trick. East can send a signal.



\vspace{0.5cm}


%--------------36 page ----------------------

Next deal shows a effort to find an entry\\

\qquad 2009/11/26 Kanagawa governer cup \#1
\begin{quote}
\crdima{ze/None}{%
  \begin{minipage}[t]{\br}
     3NT:\\OL:\d 10
  \end{minipage}}%
  {\hand{KJ1094}{73}{Q76}{11075}}%
  {\hand{832}{J9854}{10}{Q964}}
  {\hand{Q765}{AQ6}{K932}{K3}}%
  {\hand{A}{K102}{AJ854}{AJ82}}%
\end{quote}
\begin{quote}
\begin{bidding}
- \> -  \> 2\d \> 3\c  \\
P \> 3\s \> P \> 3NT\\
a.p.
\end{bidding}
\end{quote}
{\bf Remarks}\\

Opening lead is \d 10. South wins with \d J. After
cashing \s A south led \c J which wsa taken by west's
\c K. 

West led \h Q!! South won and led \c 2 to west's \c Q.
If west imaged the potion correctly, defense was easy.
West led \h J blocking \h A.

It was not a fatal wound. Hw could recover by leading spade.
But his heart lead was fatal, South nade contract,

West's \h Q is not so curious. Sometimes such a play appeared,

In another table east started bidding by weak 1NT, Contract was same 
3NT, Opening lead was \h 5. East \h Q and south ducked. West
continued heart attack, Contract failed.
%-------------------------37---------------------------
\vspace{0.5cm}

Next deals tells  how to inform of correct return .
to partner\\

\qquad 2007/4/29 Suntory cup \#4
\begin{quote}
\crdima{N/None}{%
  \begin{minipage}[t]{\br}
     Play:\\demo
  \end{minipage}}%
  {\hand{K10}{AQ98}{K103}{A1063}}%
  {\hand{AJ75}{5}{Q742}{QJ95}}
  {\hand{2}{J1064}{A085}{8742}}%
  {\hand{Q09643}{K732}{J6}{K}}%
\end{quote}
\begin{quote}
\begin{bidding}
- \> -  \> 1\d \> P  \\
P \> 1\s \> P \> 3NT\\
a.p.
\end{bidding}
\end{quote}
{\bf Remarks}\\

Oening lead is \h 5. After cashing \c K, south leads
\s 3 to dummy's \s K. Dummy leads \s 10 to west's \s J.
West convienced that partner has \d A, Eest is worring about
how can he inform of correct return to partne: 
namely that is a heart. Finaly he led \d Q  covered by \d K and \d A.
West returned diamond. East-west missed ruffing heart.

Discussion continued for a sevetral days. A friend of mine
presented a unique idea.

West leads \c Q which reveals south's singleton. Dummy drived out
west's ace which implies remaining trump. Then west leads \d 7 to east.
Since no trick can be expected in minor suits, west will find heart return.

%----------------------38-------------------
\section{basic unblocking}

\qquad 2006/2/10 Sibutani cup \#21
\begin{quote}
\crdima{N/None}{%
  \begin{minipage}[t]{\br}
     Play:\\demo
  \end{minipage}}%
  {\hand{QJ10864}{-}{AJ1076}{62}}%
  {\hand{7}{QJ10542}{9}{K10875}}
  {\hand{A9532}{K63}{842}{94}}%
  {\hand{K}{A1087}{KQ53}{AKAQJ3}}%
\end{quote}
\begin{quote}
\begin{bidding}
- \> P  \> P \> 1\d  \\
2\h \> 2\s \> P \> 3NT\\
a.p.
\end{bidding}
\end{quote}
{\bf Remarks}\\

Opening lead is \h Q. East plays \h 6(
start of encourage). South holds up.
West leads \h J at trick two. East plays \h K
for unblocking. South was glad with \h 987.
He made contract easily. 

West's second lead should be small heart.
Be sure to understand basic unblocking.

A note should be added that care of defensive 
played correctly.

South can make 3NT if he ducked twice.
At another table,south reached to 6\d after
north showed two suiters: spade and diamond.

The miss of defence is not talked by this slam.




%---------------------39------------------------
\vspace{0.5cm}
{\bf closing intermission}\\
\begin{description}
\item[no 1]{
\qquad \qquad \qquad \c 4\\
\qquad \c KQ8752 \qquad \qquad \c J103\\
\qquad \qquad \qquad \c A96\\

Oening lead is \c K. East plays \c J foe 
unblocking. South wins, When west gets a lead,
he cashes \c Q and east plays \c 10. South was
unbeliebale about winning his \c 9.

West should lead a samll at trick 2.
}
\item[no 2]{


\qquad \qquad \qquad \h xx\\
\qquad \h J1097 \qquad \qquad \h Qx\\
\qquad \qquad \qquad \h AK8x\\

Contracr is NT.
Heart suit is an only suit which threatons south.
Oening lead is \h J . West plays small heart( in
principle unblocking is prefered)
South won with ace because 4-3 break  or 5-2 with  doubleton
queen is safe. Later west leads \h Q, Of course souch ducks.
Heart attck stops.

Afer the game , Mr.stupid accused west ignored
unblocking at the first trick, But south wins it with ace,
West has a chace later ( south is carefule of avoidance play)
his \h 10 is ducked, No more attck is possible.
Fourth best is a solution even in the case J1097x.  
}
\item[no 3]{
Do you have a different lead between
K-Q-J-x and K-Q-J ?

}
\end{description}
\vspace{0.5cm}

%---------------------40 page------------
Following hand shows an opening lead with good sequence.\\

\qquad 2005/3/29 Yokohama sectional \#24
\begin{quote}
\crdima{S/vul:Both}{%
  \begin{minipage}[t]{\br}
     1NT\\OL:\h K
  \end{minipage}}%
  {\hand{A6}{95}{Q87542}{1076}}%
  {\hand{Q943}{AKQ6}{63}{K53}}
  {\hand{1072}{J10873}{9}{Q842}}%
  {\hand{KJ53}{42}{AKJ10}{AJ9}}%
\end{quote}
\begin{quote}
\begin{bidding}
- \> -  \> - \> 1NT  \\
a.p.
\end{bidding}
\end{quote}
{\bf Remarks}\\

Opening lead \h K seems an easy defense. But
actual defense is sad story.

West led second trick \h Q . East played \h 7 then
\h 3. West was wondering about his \h A ,because it might give 
south a trick if south has four cards including 10.

So he decided th lead \h 6, Bad dream comes actually.

In the postmotem east told his \h J might protectet
for the trouble. A question is  \h J is not a complete 
sequence (J-10-9). But dummy has doubleteon including 9.
So J-10-8-7 is complete. 

Anyway east should have paid attention about his 
five cards.

{\bf closing intermission}\\

At the third hand, there is a case that he discards high 
card, He  does not want to show an attitude \footnote{
An attitude signal is the most common signal in bridge. A high spot card encourages the lead of a suit, whereas a low spot card discourages.

Attitude signals are used when following suit to partner's lead, or when making the first discard in a suit. Of the three types of defensive signals, attitude
 signals have the highest priority:\\
Attitude\\
Count\\
Suit-preference} or a counting.
\index{rule of eleven}
%---------------------43-----------------
\section{rule of eleven against unblocking}

\qquad 2005/5/29 Yokohama sectional \#19
\begin{quote}
\crdima{vul:E-W}
%\begin{minipage}[t]{\br}
%     dealer:S\\vul:both
%  \end{minipage}}%
  {\begin{minipage}[t]{\br}
     3NT\\OL:\h 10
  \end{minipage}}%
  {\hand{QJ1072}{84}{Q954}{A6}}%
  {\hand{984}{A10932}{1082}{92}}
  {\hand{A65}{KQ75}{6}{K8754}}%
  {\hand{K3}{J6}{AKJ73}{QJ103}}%
\end{quote}
\begin{quote}
\begin{bidding}
- \> -  \> - \> 1NT  \\
P \> 2\c \> P \> 3\d\\
P \> 3NT \> a.p.
\end{bidding}
\end{quote}
{\bf Remarks}\\

Opening lead ia \h 10. East wind with \h Q. West 
returns the original fourth best \h 5. South plays \h J.
As west is believing south has \h K, west ducked!.
Overlooking dummy's \h 4 might effect the defense.

Fortunately this ducking was not fatal. Due to lacking winners 
south went down one.

There was an information from bidding that
south has three heart cards at most. West's heart length 
is longer or eual to east's one, So east should play \h K
at trick two in order to avoid blocking.

Remember there is an important play other than 
original fourth best.


Follong deal is same example.

%-------------------------------------43 page----------
\vspace{0.5cm}
\qquad 2006/10/27 Yotsuya league \#26
\begin{quote}
\crdima{vul:both}
%\begin{minipage}[t]{\br}
%     dealer:S\\vul:both
%  \end{minipage}}%
  {\begin{minipage}[t]{\br}
     3NT\\OL:\d Q
  \end{minipage}}%
  {\hand{A1064}{J8}{6}{Kj7543}}%
  {\hand{9}{A7642}{QJ932}{86}}
  {\hand{K8732}{KQ103}{7}{Q92}}%
  {\hand{QJ5}{95}{AK1085}{A10}}%
\end{quote}
\begin{quote}
\begin{bidding}
- \> -  \> P \> 1NT  \\
P \> 2\c \> P \> 2\d\\
P \> 3\c \> P \> 3NT \\
a.p.
\end{bidding}
\end{quote}
{\bf Remarks}\\

South promissed 16-17 HCP on 1NT. But such an invented bid
often appears. South wins opening lead \d Q with king.

South led \s Q and east won with king. East led \h K.
West encouraged wih \h 7. At this point since east led \h 3,
the suit was blocked, Althouhgh the game went down, the defense 
can not be praised. 

Fourth best is standard in four cards, But east can know south
has at least three card in heart from the bidding. Partner may
has longer heart than east. So east should  play \h Q 
after \h K.

A person who has less cards should unblock.
 
\vspace {0.5cm}
{\bf closing intermission}\\
%---------------44---------------
\begin{description}
\item[no 1]{

The case when a defenderreturns his partner's
suit.

If he has four or more cards, original fourth best is
basic. If he has three cards ( he had played the highest card
at first),
higher card is played st second,


}
\item[no 2]{ The case when east shifts to new suit.\\
If you have an honour suach  as A984 ,fourth best is basic.
If dummy and you are weak , the highest card is basic ; that is
top of nothing.  For example , 8 is played at 8765.

\index{MUD}
Even if you have promissed  secound high or fourth best for 
opening lead, you might change the rule in midle game.
MUD\footnote{
MUD refers to the order in which a defender plays three small cards. The opening lead is the middle card, followed by the higher card and then lower card, In comparison to leading "low from three small", MUD tries to convey weakness in the suit. At the same time, MUD tries to avoid being confused with a doubleton when the higher card is played at the defender's second turn.}
is also applicable. 
}

\item[no 3]{ The case KJx or K10x.
Jack or ten is led often unlike the case of opening lead.

}
\end{description}

\vspace{0.5cm}

\index{rule of eleven}
Following deal shows combination of rule of eleven and 
bidding can protect blocking.
%-----------------------44----------------

\qquad 2007/7/21 Montalt cup \#5
\begin{quote}
\crdima{S/N-S}{%
  \begin{minipage}[t]{\br}
     3NT\\OL:\d 8
  \end{minipage}}%
  {\hand{1097}{A732}{10}{KQ876}}%
  {\hand{5}{K9854}{AJ986}{102}}
  {\hand{QJ842}{6}{KQ43}{954}}%
  {\hand{AK63}{QJ10}{752}{AJ3}}%
\end{quote}
\begin{quote}
\begin{bidding}
- \> -  \> - \> 1NT  \\
2\h  \> 3\c \> P \> 3NT\\
a.p.
\end{bidding}
2\h : heart and minor
\end{quote}
{\bf Remarks}\\

Opening lead is \d 8. East wins with \d Q and returns original fourth best \d 3.South played \d 5 then \d 7 consealing \d 2. When west won with \d 9, he 
wondered why east chose \d 3. West imaged a pattern \d Q32 for east. It means
south has four card incluing king. At last west shifed to spade.

On the postmotem east could know south ha not a card higher than 8 according
to rule of eleven\footnote{
Rule of 11 is applied when the opening lead is the 
fourth best from the defender's suit.
 By subtracting the rank of the card led from 11, the partner of the opening
 leader can determine how many cards higher than the card led are held by 
declarer, dummy and himself; by deduction of those in dummy and in his own hand, he can determine the number in declarer's hand.}.
So east should lead \d K at secound trick.West who was deceided by south's
deceptive card is also acused.

At another table, the bidding and contract were same using Cappelletti\footnote{This convention is known by several names because Mike Cappelletti Sr., Fred Hamilton, and Julian Pottage (in the UK) are all credited for it. For simplicity I am just going to refer to it as Cappelletti.
Cappelletti is a defensive bidding convention after an opponent has opened 1NT. Overcaller's available conventional calls are described in web site.}.

\index{Cappelletti}
Oenind lead was \d 8. When east won with \d Q , he returned \h 6. He might
think \d 8 seems to be top of nothing.

It is not quite easy to protect blocking.

\vspace{0.5cm}

%-----------46------------------
Next deal shows a careful difense between a pair.\\

\qquad 2007/7/13 Totsuya league \#2
\begin{quote}
\crdima{vul:E-W}{%
  \begin{minipage}[t]{\br}
     3NT:\\OL:\c 2
  \end{minipage}}%
  {\hand{KJ97}{83}{AKJ107}{A8}}%
  {\hand{108652}{Q95}{4}{J1072}}
  {\hand{A}{KJ2}{Q985}{Q9854}}%
  {\hand{Q43}{A10764}{632}{K3}}%
\end{quote}
\begin{quote}
\begin{bidding}
- \> 1\d  \> P \> 1\h \\
P \> 1\s \> P \> 1NT\\
3\d \> P \> 3NT \> a.p.
\end{bidding}
\end{quote}
{\bf Remarks}\\

Opening lead is \c 2 followed by \c 8 \c Q and \c K.
South led \s 3 followed by \s 2 \s k and \s A.

East returned \c 5 carefully showing original fourth best.
West calculated 11-5 revealing south has no card higher than
5. So west played \c 10 not \c 7. When east got a lead  
through dummy's finesse, east returned \c 4, showing 5 cards.
West overtook by \c J and returned \c 2,

This defense shows careful pair avoiding to block 
with each other.

\vspace{0.5cm}

{\bf closing intermission  Which is better 4th best or 3rd best?}\\

Since the era of whist (250 years ago), fourth best(4th best)
 has been used.
This is a traditional lead. Onthe othe hand third best (3rd best)
is a new lead and apperared in late 20 centuries.
They have their own mirit and demerit. Their usage depends upun the
players. Some pair uses both simaulteneously; 4th best in NT and 
3rd best in suit contract.

An example is below.\\

\begin{quote}
\crdima{N/None}{%
  \begin{minipage}[t]{\br}
     Play:\\demo
  \end{minipage}}%
  {\hand{AJ9543}{963}{2}{1074}}%
  {\hand{-}{-}{-}{-}}
  {\hand{72}{K4}{KQ10873}{J96}}%
 {\hand{-}{-}{-}{-}}
\end{quote}

The game is defending agaist 5\d and west leads \s4,
north \s K, east \s A and south \s 6. What is your return?

Curious east concluded they can not win a spade any more,
because fourth best indicates south's doubleton. He reurned
a heart which west had called as a safty lead.

When 3rd best is used, west tells 3 cards or 5. But
nothing about an honour. Let's leave the spade and shits to 
heart.

Problem depends upon the policy of defense, not on length.

Let's change the point of view. In order to defeat 5\d ,
defender must take three tricks, Since club is ay most one,
defender must take two tricks in spade. Without hesitation
sapde must be returned, West has Jxx. Unless the spade attack
south can discard spade looser under the dummy's club.

There is a advise for you. If you wants to determine
the number of cards, you had better to adopt 3rd best.
If you wants about an honour, 4th best is for you.

It is imprtant to master the lead.
%--------------48-------8888-----------------------

\section {to avoid being endplayed}
\qquad 2007/11/24 Kanagawa Governer Cop \#24
\begin{quote}
\crdima{vul:none}{%
  \begin{minipage}[t]{\br}
     5\h:\\OL:\s A
  \end{minipage}}%
  {\hand{Q3}{KJ752}{AKJ}{AK3}}%
  {\hand{AJ10954}{6}{1062}{J52}}
  {\hand{872}{Q4}{Q832}{Q1084}}%
  {\hand{K6}{A10983}{974}{976}}%
\end{quote}
\begin{quote}
\begin{bidding}
- \> -  \> - \> P \\
2\s \> Dbl \> P \> 4\h\\
P \> 5\h \> a.p.
\end{bidding}
\end{quote}
{\bf Remarks}\\

Since north-south pair uses Levensohl
\footnote{
the basic principle behind the Lebensohl convention is to use a 2NT bid as an artificial relay, asking partner to bid Three Clubs. This can be applied in numerous auctions, and we will be covering a number of those in related articles.
Today, we concentrate on the situation in which Lebensohl is most frequently
applied, when partner's 1NT opening is overcalled by RHO.
} convention,
north thought south's 4\h is more strong than 
Levensohl. North invited to slam but
south passed as a matter of course.

Opening lead is \s A. Continuation followed.
Declerer took \h A \h K \c A and \c K and led a club.
Accident occured when east won with queen.
South made his contract easily.

In order to avoid the accidebt, west should threw queen at
second club or play \c 10 at third round. East's jack
can break the contract.


Next deal shows an example to avoid endplay.

\vspace{0.5cm}
%----------------------------49------------------
\qquad 2005/8/6 Yokohana mayor cup \#17
\begin{quote}
\crdima{vul:none}{%
  \begin{minipage}[t]{\br}
     4\h:\\OL:\c K
  \end{minipage}}%
  {\hand{AQ64}{1095}{108765}{2}}%
  {\hand{J84}{6}{KJ9}{KQJ965}}
  {\hand{K1097}{87}{Q3}{A10743}}%
  {\hand{52}{AKQJ432}{A42}{8}}%
\end{quote}
\begin{quote}
\begin{bidding}
1\c  \> P  \> 1\s \> 4\h \\
a.p.
\end{bidding}
\end{quote}
{\bf Remarks}\\
Opening lead is \c K, Next west led \s 4. East won the
finesse by \s K and retured a sapde. North's \s A won. \s 3
was led and south ruffed by high trump. Enter to dummy with 
\h 10 and ruffed out last spade by high, Enter to dummy again by \h 9
and declarer led a small diamond to ace. Sourh led a diamond from hand.
West played \d J . Alas east won by queen. All east could do is to lead a club.
Fine throw-in play!! 

If west played king and succeeded to defense,
his play wold ber rare Crocodile Coup\footnote{
The Crocodile Coup is a play in the game contract bridge. It is executed by the defense: specifically by the second hand to play to a trick. It is the play of a higher card than might seem necessary, to keep a run of honors from being blocked by a singleton honor being in the other hand with either no entry back to the remaining tricks, or having to return the lead to declarer who can promptly dispose of his losers. 
}.  You can image a crocodile opens big mouth( \d K and \d J) and
swallows partners honour. This appears ofen in many books but I 
have neve seen it before in real game.

Defenders can feel south's endplay from ruffinf out spades.
So east had better to unblock \d Q when declare led a smal from 
dummy. East can notice west has qeeen otherwise declarer would
have tried diamond finess. So east could play his king.

Defenders were stunned and the madonna sitting at south smiled with
gratitude. 

\vspace{0.5cm}
%------------------50--------------------
{\bf mission impossible "A Seasaw squeeze"}\\
\qquad 2007/3/30 Studio \#20
\begin{quote}
\crdima{vul:both}{%
  \begin{minipage}[t]{\br}
     4\h:\\OL:\h 9
  \end{minipage}}%
  {\hand{-}{A5432}{QJ10863}{84}}%
  {\hand{AQ83}{9}{AK7542}{K5}}
  {\hand{10972}{1087}{9}{Q10963}}%
  {\hand{KJ654}{KQJ6}{-}{AJ72}}%
\end{quote}
\begin{quote}
\begin{bidding}
-  \> -  \> P \> 1\s \\
Dbl \> 2\d \> P 2\h \\
P \> 4\h \> a.p.
\end{bidding}
\end{quote}

{\bf Remarks}\\
As west interpleted the bidding well, he chose  a trump
lead according to standard policy. South won by hand and
ruffed a small spade in dummy. \d Q kicked out west's king
discarding south's \c 2. 

After a long thought west led \d2 . Dummy played \d J .
East ruffed a winner. South overruffed.

\d J might be doubtful. South's original plan counts 10
winners in dummy: 5 hearts 4 diamond( \d QJ kick out AK, \d 108
are winners . ruffing small diamonds twice) and 1 club. \d 2 instead of \d J
is along this plan.

Since East ruffed a winner, his ruffing seems natural. But
another choice is possible, 

South lead \h K to collect a last heart. At this critical point
west discarded a diamond. So dummuy overtook by ace and led 
\d 10 to west's \d A. West shifted to club along the partner's signal 
\c 6. Dummy has enough entries to establish his diamond.

In postmotem. When west won \d K , \c K was an idea.
Another discussion is to discard a spade at critical point.
The professor found another methode to make contract.
He told dummy did not overtake. South ruffs spade and
establishes spade suit. 

This two ways establishing is called seasaw squeeze
\footnote{
seesaw squeeze,simply squeeze , is a technique used in contract bridge and other trick-taking games in which the play of a card (the squeeze card) forces an opponent to discard a winner or the guard of a potential winner. The situation typically occurs in the end game, with only a few cards remaining. Although numerous types of squeezes have been analyzed and catalogued in contract bridge, they were first discovered and described in whist
}.
Another name is entry exchanging squeeze.

Discussion continues. West could nake \c K by discarding 
\c 5. South is difficult to find out the position,
He might do mistake. It is an important to
invite opponent's miss.

%------------------52--------------------
{\bf mission impossible "Scissors coup"}\\

\index{Scissors coup}
\qquad 2007/7/21 Montalt cup \#22
\begin{quote}
\crdima{vul:N-S}{%
  \begin{minipage}[t]{\br}
     4\s:\\OL:\d 9
  \end{minipage}}%
  {\hand{Q984}{Q6}{J64}{AK109}}%
  {\hand{A1083}{A1085}{93}{543}}
  {\hand{5}{9732}{KQ852}{876}}%
  {\hand{KJ72}{KJ4}{A107}{QJ2}}%
\end{quote}
\begin{quote}
\begin{bidding}
-  \> -  \> P \> 1NT \\
P \> 2\c \> P \> 2\s \\
P \> 4\s \> a.p.
\end{bidding}
\end{quote}
{\bf Remarks}\\
This deal praises south's play rather than dehender's one.
Declarer often controls the communication between defenders.
Such a play is called Scissors coup\footnote{
Scissors coup (or, Scissor coup, also at one time called The coup without a name) is a type of coup in bridge, so named because it cuts communications between defenders. By discarding a card or cards either from declarer's hand or from dummy or both, declarer can stop them from transferring the lead between each other, usually to prevent a defensive ruff.
}
, another name is
no-name-coup.

Opening lead is \d 9. East played \d Q.
South wondered whether west's diamond is 
singleton or doubleton. He ducked correctly.
West returned \d K. South won by ace.
South led a heart. West ducked and dummy won with \h Q.
West won nxt heart by ace. West retured a club.

Before drawing trumps, souch cashed \h J discarding 
a diamond in dummy. This has an effect of overruff agaist west's diamond ruff
even if east has ace of trump. 

Another attention should ne noted for drawing trumps
about 4-1 break. Since west has short in diamond , he 
has a much possibility having 4 cards of trump.
As dummy has important \s 9, trump lead should be a
low card from dummy at first. 

All careful plays are worth to be rewarded.

At another table, declare won the opening lead 
immedeately. Defender could ruff the diamond.

 