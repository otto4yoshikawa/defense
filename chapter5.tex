
Even if we decide each other's signal, make it complete.
Play the discard correctly and
It is essential that partners understand it correctly.

It's easy to say in words, but not always in actual battles
not easy.

Discards can sometimes be a source of misunderstanding
One reason why defense is difficult.

\section{discard differs from follow restrictly}
\qquad 2007/4/29 Suntory Cup \#21
\begin{quote}
\crdima{W/vul:E-W}{%
  \begin{minipage}[t]{\br}
     2NT \\OL:\s A
  \end{minipage}}%
  {\hand{J762}{Q1072}{A}{KQ94}}%
  {\hand{AKQ9}{K4}{9853}{1085}}
  {\hand{103}{J98653}{K74}{72}}%
  {\hand{854}{A}{QJ1062}{AJ63}}%
\end{quote}
\begin{quote}
\begin{bidding}
1\d  \> Dbl   \> P \> 2\c \\
P \> 2\d \> P \> 2NT \\
a.p.
\end{bidding}
\end{quote}
{\bf Remarks}\\

Opening lead is \s A. Dummy was open.  West shifted to diamond .
South comes back to his hand with \c J  and led \d Q. East won with
\d K and returned \s 10. West won with \s Q and led a club.
South won and took three top diamonds and led \s 8. East discarded
(not follow) \h 5 and \h 4. The meaning of this signal differs
in the pair. Esat intended even-odd signal and west interpreted as
a encourage signal, West led his \h K . West was wrong.

\index{trump echo}
Normally a follow or discarg starts from the lowest with 
sequentially. The word discard is used for both follow and discard.

In suit contract there is a special rule for following trumps
,named as trump echo \footnote{
A trump echo is a count signal in the trump suit. It is used when following suit or when ruffing a trick. Playing high-low in the trump suit indicates an odd number of trumps (usually three). This is opposite to standard count signals, in which playing high-low shows an even number of cards.

The purpose of the trump echo is to provide partner with a count of the trump suit, especially for the purpose of getting a ruff. Trump echoes should be reserved for these situations only. (Declarer gleans too much information if you always use trump echoes to give count.)

As a corollary, playing low-high in the trump suit indicates an even number of trumps. These agreements are part of standard carding, and should not be considered special carding.

}
. When a defender has 2 or 4 cards,
he follows low and high.
If he has three, his first is middle then low.

The mechanics of trump echo is based by th fact the top is reserved for
overruff or promotion.


%-----------------------8-----------------------

\section{first discard is attitude}

When there is a promise of "first discard is attitude", a negative
attitude falls in difficult for chosing among remaining suits.\\
For example\\
\qquad 2007/4/270 Yotsuya league \#7
\begin{quote}
\crdima{N/vul:Both}{%
  \begin{minipage}[t]{\br}
     3NT \\OL:\s 10
  \end{minipage}}%
  {\hand{KJ96}{AK109}{A1065}{A}}%
  {\hand{105}{J53}{984}{K10943}}
  {\hand{A832}{Q862}{Q2}{Q76}}%
  {\hand{Q74}{74}{K732}{J852}}%
\end{quote}
\begin{quote}
\begin{bidding}
- \> 1\d  \> P \> 1NT \\
P \> 3NT \> a.p.
\end{bidding}
\end{quote}
{\bf Remarks}\\

West leads \s 10. As it doed not be accompanied by 9, east
knows that is from short suit. As he finds no suitable suit attaked,
he ducks once. South wins with \s Q and soldiers on spade twice.
East was eagerly watching west's discard. West played \h 3.
This does indicate a suit for east. West thought both minor to
be regretable. East returned \d Q which south ducked, \d J helped
declarer to make contract.

At postmotem west explained \h 3 intened a suit preference;
He should have send a clear signal. Their misunderstanding
should be recovered.

\vspace{0.5cm}

Next deal shows a discard of attitude.\\

\qquad 2004/11/12 Koezuka Cup \#5
\begin{quote}
\crdima{S/vul:N-S}{%
  \begin{minipage}[t]{\br}
     3NT \\OL:\d 7
  \end{minipage}}%
  {\hand{9}{AKxx}{Q108x}{xxxx}}%
  {\hand{K102}{Q8}{AJ87xx}{xx}}
  {\hand{QJ543}{J10xx}{-}{QJxx}}%
  {\hand{A876}{9xx}{Kxx}{AKx}}%
\end{quote}
\begin{quote}
\begin{bidding}
- \> -  \> - \> 1\c \\
2\d  \> Dbl \> P \> 2\s \\
P \> 3\c \> P \> 3NT \\
a.p.
\end{bidding}
\end{quote}

\index{intermediate}

2\d : intermediate\footnote{
Intermediate jump overcalls are more constructive than weak jump overcalls. A jump overcall at the 2-level shows a 6-card suit and 11-15 points. A jump overcall at the 3-level shows a 7-card suit and 11-15 points.
Many players consider intermediate jump overcalls to be more effective against good opponents, whereas weak jump overcalls may be more useful against bad ones. As a middle-of-the-road approach, expert Marshall Miles has noted in his book "Competitive Bidding in the 21st Century" that he prefers intermediate jump overcalls only when vulnerable. Thus, weak jump overcalls can still be played in the safer waters of white-on-white or white-on red vulnerability.}\\

{\bf Remarks}\\

Against opening lead \d 7, dummy covers with \d 8. East discards \s 5.
South lead small diamond from dummy, East did \s 3( clear encourage).
South king and west wins with ace. West refused to lead spade because
east has only three cards. He shifted to \h 8. Dummy small, east won
with 10. East returned heart and south ducked again. West who won
with \h Q. If west shifted to a spade , contract went to down one.
But west shifted to diamond. East parted a club. Declrer could
establish dummy's fourth club.

Noisy postmotem is about east's \s 3. It wsa contradict betwwen
attack and discard,

It is quite clear that east will be in trouble discarding 
among three suits An advise is to give up the suit located in
left hand, small club!. (Cassiwei's book)


 Another defense was for west to shift spade when he won with \h10.





%--------------------101--------------------------
\section{negative discard}

\qquad 2007/2/21 Yokohama Cup \#8
\begin{quote}
\crdima{W/vul:none}{%
  \begin{minipage}[t]{\br}
     3NT \\OL:\s 3
  \end{minipage}}%
  {\hand{6}{Q54}{J9875}{AQJ6}}%
  {\hand{A10932}{6}{A6532}{93}}
  {\hand{QJ85}{J109732}{-}{K74}}%
  {\hand{K74}{AK8}{KQ10}{10852}}%
\end{quote}
\begin{quote}
\begin{bidding}
P \> 1\d   \> 2\h  \> 3NT\\
a.p.
\end{bidding}
\end{quote}

{\bf Remarks}\\

Opening lead is \s 3. East \s J, south \s K. South leads \d K and \d Q.
West wins at the second trick with ace, while east discards \h 2 then \h 3.
West wanted to find an entry to east to attack spade from east.
West recognized \h 2 is discourage and \h 3 shows suit-preference.
As west led a club,south can collect 9 tricks.

Ppostmotem was very noisy. West insisted that east should have
played \h J.( \h J might be ambiguous.) Someone said \c 4 is kind.

What is the mistake about west? He believed south has \s Q.
He should have change mind, South had 7 red tricks, So defense
must be in a hurry.

About the east's explanation. He did not part \s 8 in the
earlier stage. Early discard means the negative signal of the suit.

When west gets a lead by \d A, he could succeeded the defense by leading 
\s 10, avoiding spade blocked.
%----------------102-------------------
\vspace{0.5cm}

Next deak shows negative discard again.\\
\qquad 2007/4/27 Yotsuya league \#20
\begin{quote}
\crdima{N /vul:Both}{%
  \begin{minipage}[t]{\br}
     3NT \\OL:\s 6
  \end{minipage}}%
  {\hand{A985}{AQ1065}{Q7}{AK}}%
  {\hand{KJ6}{K9432}{32}{J43}}
  {\hand{Q102}{J7}{KJ98}{10765}}%
  {\hand{743}{8}{A10654}{Q982}}%
\end{quote}
\begin{quote}
\begin{bidding}
- \> 1\h   \> P  \> 1NT\\
P \> 3NT \> a.p.
\end{bidding}
\end{quote}

{\bf Remarks}\\

Opening lead is \s 6. dummy small, Ease wins with \s Q.
East returns \s 10 believing fourth best. West overtook with
\s J and shifted to \c 3. South won at dummy and led \d Q.
East covered and south won with \d A, He led spade to ace and 
leads 13 th spade.East discarded \d 9. South \c 2. West \d 2.
South took \c K and led \d 7. East won with \d J. It was 
evident for east that west has at least 4 heart headed by king,
West discareded \h 2. East led \h J . South \h 8. West 3. Dummy
ducked!. East led club. South runs all diamonds, He made contract.

Postmotem was noisy. West should have discard club, It was a 
negative inference.

\index{odd-even signal}
In the last two sections, all four deals peoduce misunderstanding
between east and west, To investigate more study, odd-even signal and
Lavinsoule first discard are recommended.


%----------------103------------------------
\section{thirdhand follow}

You have mastered the basic of following and discarding. Here
we study the case of hornour sequence whree the order is not
upward, That is top of the sewuence.

In an example west leads opening lead. The third hand sitting steast, 
follows the top of the hornour sequence indicating he has no higher
cards and has a next card below. This is neither  an attitude not
a signal.The most simple case is: to play queen against opening lead
of king. East garentees a jack unless queen singleton.
West can choose who is the next leader

Before the talk about hornour sequence, we show a detourabout
the followings when dummy is winning. A lot of opinions appear
about attitude or count.


\begin{enumerate}
\item{ 1NT\\
\qquad \qquad \s KQ6\\
\s AJ1087 \qquad \s 93\\
\qquad \qquad \s 542\\

Contract is 1NT by south. Opening lead is \s J. Dummy plays \s K.
East follows \s 9 smoothly. Is this an attitude or a count signal?
Recently opinion is changing from attitude to count.
At second trick west wins. He keads \s 10. An accident occured by
dummy's duck. South thought \s 9 is encouraging. It was misunderstanding.
If east had ace ,he must have take ace and reurns \s9.

}

\item{ 3NT\\
\qquad \qquad \d AQJ96\\
\d K10852 \qquad \d 73\\
\qquad \qquad \d 4\\
Opening lead is \d 5. Dummy ace. East \d 7 naturally.
West denies \d 3 in east.  Later he made a mistake in
counting diamond. At postmotem west insisted no signal
in NT. East competed. He expalined if south has two,
he never play ace at the first trick. Discussion closed.

Next advise from the  recent doccument:

When declarere wins your long-suit opening lead in dummy,
with queen or a lower card, partner is supposed to
give count, not attitide. We already know partner's attitue;
he could not beat dummy's card.\\
If you discard a jack,you deny possession of rhe queen.
\begin{flushright}{
Eddie Kantar\\
Teaches advanced Bridge Defense}
\end{flushright}
}
\item{
another example\\
\begin{tabular}[t]{lll}
 \  & \h AKx       & \ \\
 \h xxxx & \ \ \ \  & \\
 \end{tabular}

West's lead is like as 4th best. Dummy ace. East's
samll is a attitude not count.



}


\item{ suit contract\\
\qquad \qquad K432(dummy)\\
\qquad \qquad \qquad (E) A95\\
Opening lead is jack. dummy plays 2.
}
\end{enumerate}


Opening lead is jack. Dummy plays 2. How does east do?
East encouraged with 9. But as south holds 8,
west's J107 lost a trick. East accused west not to
lead jack without 8. West said 9  is too 
precious. Although upside-dowm signal can resolve it,
signals are not absolute. 


%---------------106-------------------
\section{to follow honour sequece third hand}

\qquad 2004/11/23 Kanagawa Governer Cup \#25
\begin{quote}
\crdima{S /vul:N-S}{%
  \begin{minipage}[t]{\br}
     3NT \\OL:\s 3
  \end{minipage}}%
  {\hand{K862}{A94}{Q65}{885}}%
  {\hand{QJ73}{J73}{10987}{Q4}}
  {\hand{1094}{KQ52}{J3}{K932}}%
  {\hand{A5}{1086}{AK42}{AJ107}}%
\end{quote}
\begin{quote}
\begin{bidding}
-  \> -  \> -  \> 1NT \\
P \> 2\c \> P \> 2\d \\
P \> 1NT \> P \> 3NT \\
a.p.
\end{bidding}
\end{quote}

{\bf Remarks}\\


Opening lead is \s 3. Dummy plays \s K. which
card should east follow?
First of all east must decide the goodness of spade attack.
As south has two cards, spade seems to be better than heart shift.
If the promise is count, east will follow 4.
If you treat 10 and 9 are sequence of honors, you can play 10.
Stupid east picked up \s 4.

When west gets a lead,he will lead either red suit.
\s 10 is one of solutions. Be careful not to treat
9876 as an honor sequence.




%---------------107----------------
\vspace{0.5cm}
{\bf closing intermission Difficult new suit developement}\\


Remember card combination.\\
\begin{enumerate}
\item{ untouchable\\
\qquad \qquad \s A9x\\
\s Jxx \qquad  \qquad  \s Qxx\\
\qquad \qquad \s K10x\\

In the card combimation above, north-soutn can get three
tricks if defender ,whoever west or east ,starts leads \d.
If north-south start, they can get only two.
}
\item{ surrounding\\
\qquad \qquad \s A9x\\
\s Jxx \qquad \qquad \s Qxx\\
\qquad \qquad \s K10x\\
If east starts with \s 3,south gets two twicks. But
if east starts with 10, south can get only ace.
East holds dummmy's 9 between his 10 and 8. So
it is equivatent to  the sequence 1098. 
}
\end{enumerate}

\vspace{0.5cm}
Next deals asks third hand play.\\
%--------------108------------------------

\qquad 2009/11/20 Yokohama setional \#27
\begin{quote}
\crdima{S /vul:none}{%
  \begin{minipage}[t]{\br}
     3NT \\OL:\s 2
  \end{minipage}}%
  {\hand{K}{K92}{J10754}{J954}}%
  {\hand{Q432}{86}{KQ32}{Q83}}
  {\hand{109765}{Q543}{A9}{76}}%
  {\hand{AJ8}{AJ107}{86}{AK102}}%
\end{quote}
\begin{quote}
\begin{bidding}
-  \> -  \> -  \> 1NT \\
P \> 2\c \> P \> 2\h \\
P \> 2NT \> P \> 3NT \\
a.p.
\end{bidding}
\end{quote}

{\bf Remarks}\\


Opening leads is \s 2. Dummy wins with king. East follows \s 6
of which meaning is not obvicious.
\c J from duumy, west won with \c Q. He had no attractive lead in red suit,
he led \s 3. South won eith jack and made contract easily. If east followed 
\s 10 at first trick, west did not lead sapde because \s J is not in east.
If east has not \s 10, \s 5 is discouraged.  A clear signal is kind to @artner.

\section{second hand follow}
%-------109---------------------

When South leads a small card from dummy, east follows the 
lowest card under some exception. For example in NT contract,
west follows the top of sequence QJ10(9) or J109(8). If the
two subsequent case such that QJxx or J10xx, east follows the
lowest. If he wants play an honour, the second honour is 
followed. Other exceptions are under slam defense as an informative
signal or unblocking.



\vspace{0.5cm}
%--------------110------------------

Sequece of honours must be complihensive to your partner.\\

\qquad 2007/3/13 Ofuna pricipal cup ellimination  \#2
\begin{quote}
\crdima{W /vul:N-S}{%
  \begin{minipage}[t]{\br}
     4\s \\OL:\c A
  \end{minipage}}%
  {\hand{Q106}{4}{KQ!98}{8542}}%
  {\hand{73}{Q652}{AJ54}{AK10}}
  {\hand{982}{J10983}{6}{Q976}}%
  {\hand{AKJ54}{AK7}{732}{J3}}%
\end{quote}
\begin{quote}
\begin{bidding}
1\d  \> P  \> 1\h  \> 1\s \\
2\h \> 2\s \> P \> 4\s \\
a.p.
\end{bidding}
\end{quote}

{\bf Remarks}\\

Opening leada are three club tops, East follows
\c 7, \c 6 and \c Q,South ruffs. A small trump to \s Q.
South leads \h 4.  East low \h 3. South ace. South ruffs 
\h 7 in dummy. South draws three rounds of trumps.
West discarded a diamond, A small diamond leads from hand.
Of course west ducked. Dummy's king won. South ruffed last club in hand.
West was in trouble for discards. West discarded a diamond. Though west won next diamond, contract made four.

West complained east's \h J was kind indicating 
sequential honours. This does not matter for making four.

\vspace{0.5cm}

When defending any contract,suit or notrump,the
standard discard from a sequence headed by a jack or higher is
the top card. If you discard a king you deny possession of the ace.
If you discard a jack,you deny possession of the queen.
\begin{flushright}{
Marshall Miles\\
All fifty two cards}
\end{flushright}
%--------------111-------112----8888---

\vspace{0.5cm}
{\bf closing intermission}\\

To defeate abvoe deal, a diamond ruffed by east is simple.
How can west find the diamond ? There are several methodes below:

\begin{enumerate}
\item{ club discouraged.\\
East discourages with 6 as an honest message. If east plays \c 7 at first and
next plays \c 9 , west can not find \c 6.

}

\item{ suit-preference\\
At above cases, west notice \c Q in south or a strenge circumstance,
West might take \d A with a little chance.
}
\index{odd-even signal}
\item{ odd-even signal\\
Our hope is beyond the signaling. This csae happens to match idd-even
signal. \c 6 means non-comeon and low rank suit shift,

Odd-even-first-discard is also avilable for investigaters.
}



\end{enumerate}

By learning too many signals, partnership might be confused.
Discard has a basic rule that a card can tell sigle message,
Two messages can not be given by a discard.\\

One card cannot be played and mean tho things at the same tomt.
\begin{flushright}{
Mike Laurence\\
Judgement at Bridge}
\end{flushright}



%--------------113------------
\vspace{0.5cm}
If you admit above, we give next advice.
The purpose of signal is to attack on the suit.
There is a special promise for attacing rhe suit
which declarer has played. That is a concept of
counter-attack. It rarely happens. You had better
to use it.

\vspace{0.5cm}
Next deal occures missunderstanding\\
\qquad 2006/8/5 Yokomama Mayor(match point) \#4
\begin{quote}
\crdima{N /vul:Both}{%
  \begin{minipage}[t]{\br}
     3NT \\OL:\h 10
  \end{minipage}}%
  {\hand{AQJ62}{A5}{98}{AK86}}%
  {\hand{K1097}{Q1097}{K32}{J3}}
  {\hand{43}{842}{J1076}{Q954}}%
  {\hand{85}{KJ63}{AQ54}{1072}}%
\end{quote}
\begin{quote}
\begin{bidding}
P  \> 1\s   \> P  \> 1NT \\
P \> 3NT \> a.p.
\end{bidding}
\end{quote}
{\bf Remarks}\\
%----------------113--------------------
South won the opening lead with \h J. Spade finesse with \s Q.
\d 8 from dummy. Which card does east follow?
Esat followed \d J ,top of the sequence.South plaed \d Q.
West was in worry to take or duck.
\d J did not conveys east's ideaa.
West was afraid of 3 diamond tricks if he took. So he ducked.
This was a fatal error in match point game.
Easr explained \d J was a couter attack. But his opinion was rejected.

\vspace{0.5cm}

%--------------114-----------------
\qquad 2007/12/22 Blue ribon  \#25
\begin{quote}
\crdima{S /vul:N-S}{%
  \begin{minipage}[t]{\br}
     3NT \\OL:\c 5
  \end{minipage}}%
  {\hand{KJ53}{K754}{93}{874}}%
  {\hand{42}{A6}{J6542}{J965}}
  {\hand{Q10986}{J832}{K7}{K3}}%
  {\hand{A7}{Q109}{AQ108}{AQ102}}%
\end{quote}
\begin{quote}
\begin{bidding}
- \> -   \> -  \> 1\d \\
P \> 1\h \> P \> 2NT \\
P \> 3\h \. P \> 3NT \\
a.p.
\end{bidding}
3\h :shows both majors
\end{quote}

{\bf Remarks}\\

Opening leadis \c 5. East \c K.South \c A. Souyh
leads \h Q from hand . West wins with \h A. He has
no safty lead, There are many cases of loss when a
new suit is developing. So he led heart.
Dummy small. East wins with jack. In nornal sense 
east will lead club. But east's mistake is a heart return.
At last west was forced to lead club from his hand.
East's heart return was too passive,

A following an honour sequence might invite a misunderstanding.
Next deals are about an example.

\section{application example of follow honour sequece}
%--------------115-------------------------

\qquad 2004/11/13 Koezuka Cup \#5
\begin{quote}
\crdima{N /vul:N-S}{%
  \begin{minipage}[t]{\br}
     3NT \\OL:\s 2
  \end{minipage}}%
  {\hand{AK}{AK9x}{10xxxx}{KJ}}%
  {\hand{10872}{108}{QJ8x}{10x2}}
  {\hand{J653}{QJxx}{AK}{xxx}}%
  {\hand{Q9x}{xxx}{xx}{AQxxx}}%
\end{quote}
\begin{quote}
\begin{bidding}
- \> -   \> P  \> P \\
P \> 1\d \> P \> 1NT \\
P \> 3NT \> a.p.
\end{bidding}
\end{quote}

{\bf Remarks}\\

At one table opening lead is \s 2. Dummy's ace wins.
After cashing \c K \c J, a small diamond from dummy. East wins with king 
and returns \s 3 pushing up the dummy's king.
South leads diamond again . East wins. East returns spade. The end.

If south has not \s Q , his play would overtake club honour hoping
3-3 break. East should return a heart hoping 10 in partner's hand.

A little technique in above deal.
When east plays \s 3 and south \s 9, west's \s 10 means the top
of sequence that is : I have no higher card than 10. This olay
is introduced in a book: as Opening leader's remaining equals.\\
\qquad \qquad 73\\
J9865 \qquad \qquad A102\\
\qquad \qquad KQ4\\

Opening lead is fourth best. East wins with ace and returns 10.
South king. West plays J. It shows the location of queen.\\

When either declarer or partner is winning the second round of
the suit you have led, and you remain with equals, following
sui with highest equal denies a higher card.\\
\begin{flushright}{
Eddie Kanter\\Teaches Advanced Bridge Defense}
\end{flushright}





%-----116-----------------

\vspace{0.5cm}
{\bf closing intermission dentist coup}\\

\index{dentist coup}
At another table east bid a double against 1\d. South ia the
professor. When he saw the opening lead, he noticed east has 4-4 majors
and 3-2 or 2-3 minors, all missing hornors in east. By cashing \c K and
\c J, he found 3-3 break. After he cashed \h A \h K \s K, he led a diamond.
East won 4 winners \d A \d X \h Q and \h J , but remains black cards.
South came back by stepping stones.


This extraction of a defender's safe exit card is called the Dentist's Coup. 
Declarer plays as if he has X-rayed an opponent's hand

\vspace{0.5cm}
An application example about following honour sequence(
camouflage)\\

\qquad 2007/8/4 Yokohama mayor cup \#8
\begin{quote}
\crdima{W/vul:none}{%
  \begin{minipage}[t]{\br}
     3\h \\OL: \d K J
  \end{minipage}}%
  {\hand{J83}{6}{A10865}{AQ63}}%
  {\hand{K52}{J9}{KQJ42}{742}}
  {\hand{AQ107}{Q875}{97}{J105}}%
  {\hand{964}{AK10432}{3}{K98}}%
\end{quote}
\begin{quote}
\begin{bidding}
P \> 1\d  \> P \> 1\h \\
P  \> 2\c \> P \> 3\h \\
a.p.
\end{bidding}
\end{quote}
{\bf Remarks}\\

South won the opening lead with dummy's \d A. He cashed 
\h A and \h K ,jack fallen from west. He planed to discard
a spade under 4th club. He saw \c J at the first round.
South changed mind and led trump. He lost two trumps anf three spades.
East's deception follow leads to a good result.


\section{failure by unnecessary signal }


%------------------------118-----------------

\qquad 2004/9/18 Takamatu Cup \#4
\begin{quote}
\crdima{N/vul:Both}{%
  \begin{minipage}[t]{\br}
     6\c \\OL:\s J
  \end{minipage}}%
  {\hand{Qxx}{AKxxx}{K}{xxxx}}%
  {\hand{Jx}{QJx}{Q10xx}{Q109x}}
  {\hand{1098xxx}{10xxx}{xxx}{-}}%
  {\hand{AK}{x}{AJxxx}{AKJxx}}%
\end{quote}
\begin{quote}
\begin{bidding}
- \> 1\h  \> P \> 2\d \\
P  \> 2NT \> P \> 3\c \\
P \> 4\c \> P \> 4\d \\
P \> 4\h \> P \> 4\s \\
P \> 5\c \> P \> 6\d \\
a.p.
\end{bidding}
\end{quote}
{\bf Remarks}\\

Opening leads are same \s J in both tables.
But east's play produced a big diference. At a table \s x and 
at another table \s 10.

South won with \s A and \c A tells bad break.
\s 10 gives an information that  east has not high cards .
South took two spade tops ,and two diamods and two heart tops.
He did not cast \s Q, He made contract by throw-in into west.

At another table \s Q was ruffed by west.

This deal teaches us not to send unnecessary signal.


%------------------119-----------------

\vspace{0.5cm}

This deal relates to give deaclarer an information.\\

\qquad 2006/1/6 Asahi newspaper Cup \#20
\begin{quote}
\crdima{vul:Both}{%
  \begin{minipage}[t]{\br}
     4\h \\OL:\d 6
  \end{minipage}}%
  {\hand{K876}{9}{Q52}{Q10975}}%
  {\hand{Q105}{10643}{K1086}{32}}
  {\hand{AJ9}{2}{J9743}{KJ84}}%
  {\hand{432}{AKQJ875}{A}{A6}}%
\end{quote}
\begin{quote}
\begin{bidding}
- \> P  \> P \> 4\h \\
a.p.
\end{bidding}
\end{quote}
{\bf Remarks}\\


South wins the opening lead with ace. He runs his trumps.
When 7 remaining cards, dummy holds \s K87 \c Q1097.
East is worry to pick up a card from  \s AJ9 \d 9 \c KJ84.
South has shown  perhaps 9 winners. East.a veteran madnna, 
discarded \s 9. South did not miss it. He led spade twice and
made \s K a winner. \s 9 was a valuable present to south.

At another table,bidding and contracy were same. Opening lead is \c 3.
Followed by \c 10 \c J and \c A. Souuth runs trumps but east never threw a
spade. West knows the location of \s A. Ween south led a small spaden
west played 10. He planed west not in endplay. Dummy played king,
Contract failed.

\vspace{0.5cm}
{\bf mission impossible strip squeeze}


\qquad 2005/12/10 BTAM cup. \#9

\begin{quote}
\crdima{E/vul:N-S}{%
  \begin{minipage}[t]{\br}
     6NT :\\OL:\h Q 
  \end{minipage}}%
  {\hand{K9}{AK10}{K73}{AJ943}}%
  {\hand{107542}{3}{1095}{10765}}
  {\hand{AQ3}{QJ9854}{J42}{4}}%
  {\hand{J86}{762}{AQ86}{KQ2}}%
\end{quote}
\begin{quote}
\begin{bidding}
-  \> -  \> P \> 1\d \\
P \> 2\c \> 2\h \> Dbl \\
P \> 4NT \> P \> 5\s \\
P \> 6NT \> a.p.
\end{bidding}
\end{quote}
{\bf Remarks}\\

East leads \h Q . As north had opened his hand , he became dummy.
North won with ace. When south cashes 5 clubs and lucky four diamonds
east fall in squeeze. North leaves \s K and \h K10.
If east leaves \s A alone, he is thrown and leads heart from him.
\index{strip squeeze}
This squeeze is called as strip squeeze. They says a squeeze 
without ducking or thrust and party.

This is an interesting hand for discarding.
