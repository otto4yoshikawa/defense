I have written this book from the interesting deals which I colleced.
There are several deals which dose not belong to any chapters.
They are showed in this chapter as an appendix.

\section{to double Stayman two club}

\qquad 2006/4/21 Yotsuya League \#19
\begin{quote}
\crdima{S/N-S}{%
  \begin{minipage}[t]{\br}
     3NT\\OL:\c 7
  \end{minipage}}%
  {\hand{A84}{A1082}{9832}{63}}%
  {\hand{32}{J7543}{J7}{AJ97}}
  {\hand{QJ65}{9}{1054}{KQ1085}}%
  {\hand{K1097}{KQ6}{AKQ6}{43}}%
\end{quote}
\begin{quote}
\begin{bidding}
- \> -  \> - \> 1NT  \\
P \> 2\c \> Dbl  \> 2\s \\
P \> 2NT \> P \> 3NT \\
a.p.
\end{bidding}
\end{quote}
{\bf Remarks}\\

After 1NT open, some artificial convension such as Stayman or Jacoby transter
\footnote{
The Jacoby transfer, or simply transfers, in the card game contract bridge, is a convention in most bridge bidding systems initiated by responder following partner's notrump opening bid that forces opener to rebid in the suit ranked just above that bid by responder. For example, a response in diamonds forces a rebid in hearts and a response in hearts forces a rebid in spades. Transfers are used to show a weak hand with a long major suit, and to ensure that opener declare the hand if the final contract is in the suit transferred to, preventing the opponents from seeing the cards of the stronger hand.

The use of the 2? and 2? (and often 2?) responses to an opening 1NT bid as transfers is one of the most widely employed conventions in the game. Less commonly, partnerships may agree to use transfer-style bids in a variety of other situations.} are often used. By doubling these artificial bid, defenders direct his 
parter to attack this suit.

This was a fantastic example. Opening lead is \c 7. East won with \c Q. South followed with \c 2. East returned \d 4. That's all.

At postmotem, east said "My double garantees both length and hornours.
If you have any hornour, lead the hornor. I placed it at south.".  
West explained " Fourth best is best. It can catch south's Q10x.".
Conclution is remote. Surely \c 7 is incomplete, But even if south has
\c A, east's club return is considerable.

Recently non balanced hand such as 6-3-2-2 or 4-4-4-1 appears in INT open.
Therefore I propose double might be more flexible. No to double may
gives the opening leader an infomation.

\vspace{0.5cm}

Next deal again againt Stayman.\\

%----------------------page 158-------------------
\qquad 2005/2/9 NEC \#29
\begin{quote}
\crdima{W/Both}{%
  \begin{minipage}[t]{\br}
     3NT\\OL: \c 2
  \end{minipage}}%
  {\hand{J864}{A643}{Q1043}{7}}%
  {\hand{K1095}{J1075}{7}{A842}}
  {\hand{A32}{92}{J52}{KJ865}}%
  {\hand{Q7}{KQ8}{AK986}{Q103}}%
\end{quote}
\begin{quote}
\begin{bidding}
P \> P  \> P  \> 1NT \\
P \> 2\c  \> Dbl \> 2\d  \\
P \> 2NT \>  P \> 3NT \\
a.p.
\end{bidding}
\end{quote}
{\bf Remarks}\\

Opening lead is \c 2. East wins with \c K, He returns \c 6.
If south played 10 , he could make contract. But he played \c Q.
The meaning of double may be different on the case of before.

At the last previous deal south bids 3NT without stopper.
There is a convinient convention to confirm a stopper.
INT opener bids the pass if he has the stopper even if he has 
major suit, Partner relays by redouble and south answer his
major. If south has not a stopper, he answers at once.




\section{to doble other artifitial bid}

\index{RKCB}
Other artifitial bids such as Sprinter or respose for
RKCB ,   are boubled for lead showing.
There are some careful points for the high level double.
\index{Sprinter}
\begin{enumerate}
\item{ Sprinter\footnote{
splinter bid is a convention whereby a double jump response in a side-suit 
indicates excellent support (at least four cards), a singleton or void 
in that side-suit (but preferably not the ace or king), and at least 
game-going strength.}\\
The stupid player holds below and\\
 bids sprinter(1D-1H-3S):\\
\s x\\
\h AKQx\\
\d KQJ10xx\\
\c Ax\\
Left hand opponent: double. Partner 4\h . Next 4\s.
Finally 4\s XX and double made. \\
Unforgettable accident. They should bid any red suit.

}
\item{ storong 2\c\\
Recently strong 2\d (18-19 ballanced hand) appears.
An opponent forgets to double:\\
2\d-P-2\s-? with the hand below\\
\s AKQ52\\
\h Q2\\
\d 76543\\
\c 3\\
As he passed, they reached at 4\h and made 4.
If he did double, his partner would bid good 
sacrifing 4\s.



}
\item{ 1NT-P-3NT-P(E)\\
East passed after a short thinking. East and west held tha hand below:\\
West:\\
\qquad \s 8\\
\qquad \h 10986\\
\qquad \d 64\\
\qquad \c 876532\\

East:\\
\qquad \s AKQJ107\\
\qquad \h Q42\\
\qquad \d J73\\
\qquad \c 9\\

West's opeing lead was \h 10. South made contract straightly.

East accused west not to lead a spade. But the person to be
blamed is east. East should have doubled againt 3NT. This
double demands an abnormal opening lead. West might find \s 8.

}
\index{Lightner double}
\item{ Lightner double\footnote{
The Lightner double is a conventional double in bridge, used to direct the 
opening lead against slam contracts. It was devised by Theodore Lightner.
The Lightner double is a call made by the partner of the player who will make the opening lead. It asks for an "unusual" opening lead. The opening lead is often crucial to the play of the hand, and the right opening lead is often the only chance for the defenders to defeat the contract. The doubler will most often have a void in a side suit, or sometimes AQ or KQ in the suit bid by the dummy. The partner is expected to find the correct lead, which might be unusual from his viewpoint; in any case, he should not lead a trump. The most common interpretation is to lead the first suit (other than trumps) bid by the opponents.
}\\
There is a famouse convention named  Lightner double against slam contract.
This demands to lead dummy's first suit. Details are omitted.
}

\end{enumerate}




%----------------------page 161-------------------
Reckless double is punished.\\

\qquad 2005/10/29 Yokohama BC cup \#10
\begin{quote}
\crdima{S/Both}{%
  \begin{minipage}[t]{\br}
     6\h X \\OL: \s 9
  \end{minipage}}%
  {\hand{863}{AJ65}{A632}{K10}}%
  {\hand{72}{KQ42}{Q854}{942}}
  {\hand{K5}{-}{KJ1087}{QJ7653}}%
  {\hand{AQJ1094}{108732}{-}{A8}}%
\end{quote}
\begin{quote}
\begin{bidding}
P \> P  \> P  \> 1\s \\
P \> 2\d  \> P \> 2\h  \\
P \> 4\h \>  P \> 6\h \\
Dbl \>a.p.
\end{bidding}
\end{quote}
{\bf Remarks}\\

If west did no bid, south plays to draw normally.
Ace cash or jack toward dummy. But greedy west doubled 
telling king and queen in his hand. 

South leds \h 10 from hand. Double finesse succeeded
reducing in one looser of trump.


\section{Journalist lead}

Journalist leads\footnote{

 The method is designed to solve some problems with traditional agreements 
regarding opening leads. It bears some resemblance to Rusinow leads 
but differences exist. Journalist leads were advocated and publicized 
in 1964-65.}
\index{Bridge Journal}

 are an opening lead convention .
Inprovement on the opening lead has been discussws in any time.
Around 1960, many convensions were contributed on the Bridge Journal.
Ace from ace and king is one of them. It might be regarded as a
standard convension.\\

%----------------------page 162-------------------
\qquad 2004/4/10 Japan league \#3
\begin{quote}
\crdima{E/N-S}{%
  \begin{minipage}[t]{\br}
     2NT:\\OL:\d A
  \end{minipage}}%
  {\hand{KQJ653}{7}{8}{KQ953}}%
  {\hand{987}{AKJ9873}{6}{107}}%
  {\hand{A104}{102}{KQJ1053}{J6}}%
  {\hand{2}{Q64}{A9742}{A842}}%
\end{quote}
\begin{quote}
\begin{bidding}
- \> - \> P  \> P \\
3\h  \>  P \> 3NT \> a.p.\\
\end{bidding}
\end{quote}
{\bf Remarks}\\
\index{Journalist Lead}
As the Journalist Lead appears rarely, they forgets
thwe promise and caurse a trouble.

West's opening lead is \H A. East followed \h 2.
West doubted east's sigleton or 3 cards, He believed
his partner forgot to send signal. It was a lucky
for west to shifted spade in order to find entry
for heary return. East won with ace and returned diamond!!

East said he has a slightest idea west has a king.
Even he forgot it, by seeing dummy's spade, he had to
find an west's emergency on spade. Inference to heart.

Another example of Journalist lead is queen from KQ108.
Queen denamds to drop jack. There is a west who forgets
the convension.\\
\qquad \qquad xxxx\\
\qquad KQ109 \qquad Axx\\
\qquad \qquad Jxx\\

West led queen. East won with ace. He shifred to other suit.




\section{last story}

This is a last story starting at the chapter count.
To enjyy or to brush up your bridge, BBO (Basic Bridge On-line)
or AI application increase the variety of them. Beleive your
study to brush up your bridge. Two examples.\\

%----------------------page 163-------------------
\qquad 1994/8/30 Yokohama \#9
\begin{quote}
\crdima{S/E-W}{%
  \begin{minipage}[t]{\br}
   6\c / 6NT:\\OL: \h 10
  \end{minipage}}%
  {\hand{AQJ7}{K62}{A5}{K1098}}%
  {\hand{K86}{103}{Q1062}{J754}}
  {\hand{432}{98754}{K873}{2}}%
  {\hand{1095}{AQJ}{J94}{AQ63}}%
\end{quote}
\begin{quote}
\begin{bidding}
- \> -  \> -  \> 1\c \\
P \> 1\s \> P \> 1NT \\
P \> 2\d  \>  P \> 2\s \\
P \> 2NT  \>  P \> 3\c \\
P \> 4\c  \>  P \> 4\h \\
P \> 4NT  \>  P \> 5\h \\
P \> 6\c  \> a.p.\\
\end{bidding}
\end{quote}
{\bf Remarks}\\

At 6\c  table they reviewed and confirmed before 
opening lead \h 10. South intuitively understood
bad club break. He planned agaist it and made the
6\c.

At 6NT table south failed to guess club and went down one.

At postmotem they talked about the luckyness of break.
They did not prosecute the reason failed. South in 6NT
took finesse spade. If he deked a diamond before cashing
three heart, he awared west's doubleton heart. It might
suggest west has longer clubs. Such an analisys is
wecome.

\vspace{0.5cm}

%----------------------page 164-------------------
\qquad 2009/7/1 Studio \#24
\begin{quote}
\crdima{W/None}{%
  \begin{minipage}[t]{\br}
     3NT:\\OL:\s K
  \end{minipage}}%
  {\hand{953}{7543}{AQJ5}{A2}}%
  {\hand{KQJ642}{J62}{8}{863}}
  {\hand{8}{K108}{K10743}{Q1074}}%
  {\hand{A107}{AQ2}{962}{KJ95}}%
\end{quote}
\begin{quote}
\begin{bidding}
2\s \> P  \> P  \> 2NT \\
P \> 3NT  \> a.p.\\
\end{bidding}
\end{quote}
{\bf Remarks}\\

\index{repeated squeeze}
West's opening lead is \s K. If south ducks twice, east worries about
discarding.  A diamond is obvicious, Whichever caerd east discards, south 
will attack the suit. It seems like a repeated squeeze.

At the table south ducked once watching a diamond discard from east.
South attacked diamond with finessing and testing even break.
Both testings failed so he went down one.

At the postmotem 13th card of heast was talked, The postmotem�@was
closed after there was a chance for south by establishing heart.

Next day, madonna announced she can thward the heart plan by
throwing her king under the ace. \h J is an entry to west.
\index{Deschapelles coup}
Such a coup has  name as Deschapelles coup
\footnote{
In bridge, the Deschapelles coup is the lead of an unsupported honor to create an entry in partner's hand; often confused with the Merrimac coup, the lead of an unsupported honor to kill an entry in an opponent's hand.

This sacrificial play was invented by Alexandre Deschapelles, a 19th-century French chess and whist player.}
.

Though details are omitted, south can establish 13 th heart by
careful plays using a plenty entries to dummy.

South can prform more easily by ducking twice in spade  unless east
does aware Deschapelles coup.
