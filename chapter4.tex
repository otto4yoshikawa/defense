%chapter 4 defender's signal--------------
There are three kinds of signal; attitue,couunt and suit-preference.
Many people promises about signals. Consequently
confusion often occurse , First of all we will explain 
those signals( standard ones)

\section{basic discard is attitude}

\qquad 2005/2/9 NEC cup \#25
\begin{quote}
\crdima{S/vul:Both}{%
  \begin{minipage}[t]{\br}
     3NT :\\OL:\c K
  \end{minipage}}%
  {\hand{J9864}{KQ}{AQJ}{983}}%
  {\hand{53}{743}{10862}{AKJ4}}
  {\hand{AQ107}{J102}{743}{652}}%
  {\hand{K2}{A9865}{K95}{Q107}}%
\end{quote}
\begin{quote}
\begin{bidding}
- \> -  \> - \> 1\h \\
P  \> 1\s \> P \> 1NT \\
P \> 3NT \> a.p.
\end{bidding}
\end{quote}
{\bf Remarks}\\

This is a popular 3NT. Opening lead is \c K. .East plays 
\c 2. West recognizes the discourarge signal denies \c Q. West
shifts to \d 2. South wins with \d Q. After taking \h A and K,
south leads a \s 4 from dummy. 

Don't sleep east! Although "second hand low" guides a low
spade, east must take with ace. Otherwise declarer collects
nine tricks adding \s K (1*5+3 +0). Whoever has \d K ,three
diamond tricks are sure,

East won with ace and returned a club. Defense is succeeded.

Attitude is the most basic signal. Here we give two notes:
One is the fact "high low" is relative relation.
Another is tha fact partner has nothing other than that.
For example bidding : 1\d - 1\h - 1NT . West leads \s Q
from QJ76. Dummy shows \s 8432 . East plays \s 5 and south
\s 9. \s 5 is the smallest one but east has \s K5 doubleton.

\vspace {0.5cm}
%----------------------79------------------------
\qquad 2005/11/4 Naniwada publish \#14
\begin{quote}
\crdima{W/vul:none}{%
  \begin{minipage}[t]{\br}
     3NT :\\OL:\h K
  \end{minipage}}%
  {\hand{AJ852}{832}{Q86}{97}}%
  {\hand{1074}{AKJ5}{J102}{A103}}
  {\hand{Q93}{94}{753}{K8642}}%
  {\hand{K6}{Q1076}{AK94}{QJ5}}%
\end{quote}
\begin{quote}
\begin{bidding}
1\c  \> P  \> P \> Dbl \\
P  \> 1\s \> 2\c \> 2NT \\
P \> 3NT \> a.p.
\end{bidding}
\end{quote}
{\bf Remarks}\\

Againt west's opening lead \h K, east plays  \h 9.
In usually the attitude picks up \h 4 as west has not queen.
East was in misunderstanding the case of suit contract.
West continues heart and gave two heart tricks to south.

There is a pair who has promised  a signal for count when
king is led. They use both signal switching on case.
If the \h 9 means number of card ie even, west is in
choice, If he think it is doubleton, he does not take ace.
If he think four cards, he cashes ace againt doublton in south.

In suit contract, only when east wants to ruff, he sends
high low signal,

%-----------------80-------------------
\section{doubleton sign or not}
\qquad 1996/10/5 Hagiwara Cup \#29
\begin{quote}
\crdima{S/vul:Both}{%
  \begin{minipage}[t]{\br}
     5\c :\\OL:\h A
  \end{minipage}}%
  {\hand{Q10543}{102}{942}{632}}%
  {\hand{J2}{AKQJ864}{7}{J109}}
  {\hand{AK876}{53}{J853}{54}}%
  {\hand{9}{97}{AKQ106}{AKQ87}}%
\end{quote}
\begin{quote}
\begin{bidding}
- \> -  \> - \> 1\c \\
3\h  \> P \> PT \> 4\d \\
P  \> 5\c \>  a.p.
\end{bidding}
\end{quote}
{\bf Remarks}\\

Against west's opening leads \h A and K, east sent doubleton
signal, 5 and 3. West thought since south has 6-5 cards in minor
suits, The only hope is partner's overrff by \c 8.
As east expected west's spade shift, he was astonished by
third round heart. Dummy's \c 6 allows to discard a spade,
The contract went to down one since south did not take diamond's
finess.

In noisy postmotem, west accused to partner 
"Don't send doubleton signal without overruffing".
A spade shift might be safe,
%-----------------------81----------
\vspace{0.5cm}\\
Next deal also asks doubleton signal.\\

\qquad 2007/11/23 Kanagawa governer cup \#2
\begin{quote}
\crdima{N/vul:E-W}{%
  \begin{minipage}[t]{\br}
     4\s :\\OL:\d K
  \end{minipage}}%
  {\hand{63}{AJ63}{A10}{J10754}}%
  {\hand{82}{Q762}{KQJ63}{982}}
  {\hand{K974}{K1098}{75}{Q63}}%
  {\hand{AQJ1052}{5}{9842}{AK}}%
\end{quote}
\begin{quote}
\begin{bidding}
- \> P  \> P \> 1\s \\
P \> 1NT \> P \> 2\s \\
P \> 3\s \> P  \> 4\s \\
a.p.
\end{bidding}
\end{quote}
{\bf Remarks}\\

\index{Drury convension}
As N-S uses 
Drury convension\footnote{
Drury is a conventional 2 response by a passed hand after partner opens 
1 or 1 in third or fourth seat. The 2 bid is artificial, 
showing a limit raise with 10-12 support points and 3+ card support.

Playing Drury, responder can keep the bidding low at the 2-level
 instead of making a jump raise. Since many partnerships open 
light in third or fourth seat, responder wants to bid conservatively 
opposite a weak hand.},
dummy can expect at least two trump cards.

Opening lead is \d K. Dummy wins with \d A and leads \d 10.
East followed with 5 and 7, not telling doubleton signal.
East expects \s 10 in west, promoting his 9. What is west's lead?
If west leads a small diamond, dummmy will ruff it. Esat overruffs.
\s K anf \d J will defeat the contract.

If west had sent doublteon signal, dummy might not to ruff 
the small diamond. It might be hard to defense.

In real game west shifted to heart. The defense failed.
West had a slightest idea about avobe defense.
 


 

%---------------83--to 84--------


\section{count-signal}

\index{even-odd}
Even when a pair has an agreement not to use count signal on the 
case of follow, they send it in next cases below:\\
Dummy has long suit and dummy has no other entry. 

We send even-odd signal in order to tell the correct timing
of cutting off to the partner. By counting the parity,
partner can estmate number of cards in hand, 

In real game there request difficult judgement. For example\\
\qquad \qquad KQ109x\\
\qquad Axx \qquad \qquad Jx(x)\\
\qquad \qquad xx(x)\\

When south leads x . West plays x. East tells count.
But Jx is valuable for discarging J. It leaves west's
judgement and south's missplay,

Another example\\
\qquad \qquad \d Q98654\\
\qquad \d A7 \qquad \qquad \d J1032\\
\qquad \qquad \d K\\

South leads K and west wins A. What is east's card?
East does not know the location of 7. Dummy has plenty entries.
\d J may be a loss. East plays \d 3. West's conclution is
east has odd. \d 2 in south.

Some players might change their signal among suit contrant and NT.

%---------84--------------------------
\vspace{0.5cm}

In high level contract an opening leader promises that
A from AK requests attitude ,while K from AK requests count.
(High level may include four.) 

\qquad 2007/9/7 Yamada 30K points \#27
\begin{quote}
\crdima{E/none}{%
  \begin{minipage}[t]{\br}
     6\h  \\OL:\s K
  \end{minipage}}%
  {\hand{Q5}{AKQJ6}{942}{AK4}}%
  {\hand{AKJ982}{94}{Q863}{5}}
  {\hand{10643}{5}{AJ1075}{J82}}%
  {\hand{7}{108732}{K}{Q109763}}%
\end{quote}
\begin{quote}
\begin{bidding}
-  \> -  \> P \> P \\
2\c \> Dbl \> 3\s \> 4\h \\
4\s \> P \> P \>5\c \\
P \> 5\h \> 5\s \> 6\h \\
a.p.
\end{bidding}
\end{quote}
{\bf Remarks}\\

Against the opening lead \s K, east plays \s10
according to promise of count signal.
West knows east has 4 or 2 cards. So he judges
south's singleton. So he attcked the diamond.
It succeeded the defense,

Although it seems a easy problem, a difficult
problem is to determine which signal west use:
A for attitude or K for count ,before dummy is
open.

%-----------------85-------------
\section{count-signal application example}

\qquad 2005/7/22 Montalt \#22
\begin{quote}
\crdima{S/E-W}{%
  \begin{minipage}[t]{\br}
     4\h \\OL:\s K
  \end{minipage}}%
  {\hand{AKQJ}{106}{J72}{J865}}%
  {\hand{8532}{J}{AQ1096}{KQ9}}
  {\hand{10976}{84}{K83}{A1032}}%
  {\hand{4}{AKQ97532}{54}{74}}%
\end{quote}

{\bf Remarks}\\

A call of 4\h closed the bidding. Opening lead is \c K.
East plays \c 3. West continued \c 9. East wins with ace.
Esat shifted to diamond, A player's choice is \d 3 ( unknown 
3rd best or 4th best). West wins. He must in guess for next lead:
club or diamond.
Another player invented \d K. He thought partner tells count
if \d K wins, In signal, first trick uses attitude , count is next.
But he believed partner's count instantly. West followed \d 6. Although \d 5
is missing yet , it must be the lowest one telling odd number of
cards. Otherwise big card such as 10 would appear. East found
diamond lead, Fine defense with easy understanding.

There is a famous saying:

The count signal usually applies when declarer or
dummy is leadind a suit. I may also be used when a
defenderfirst breaks a suit or on a first discard of
a suit if it is obvious that count,not attitude, is
of prime importance.
\begin{flushright}
{Kif Woolsey\\
Moddern Deffensive Signaling}
\end{flushright}

%--------------86-------------

\qquad 2005/11/4 Naniwada publish \#3
\begin{quote}
\crdima{S/E-W}{%
  \begin{minipage}[t]{\br}
     5\d  \\OL:\c Q
  \end{minipage}}%
  {\hand{Q732}{AK873}{K}{987}}%
  {\hand{A1096}{954}{3}{QJ1042}}
  {\hand{KJ54}{J1062}{J10}{AK5}}%
  {\hand{8}{Q}{AQ9876532}{63}}%
\end{quote}
\begin{quote}
\begin{bidding}
-  \> -  \> - \> 4\d \\
P \> 5\d \> a.p.
\end{bidding}
\end{quote}
{\bf Remarks}\\

In every defense E and W must have same line of defense.
This deal requires correct sequence of cashing winners 
when dummy is open.

Opening lead is \c Q. East plays \c5 which is odd count 
coincidently. South plays \c 6 under camouflage. West
gets the location of \c K and A. West cased \s A and east
played \s5. South \s 8. Since west placed \c 3 in east and led a spade,
south made contract. Count signal is worth to study.




%-------------87-----------------------
\section{present count}

\qquad 2007/11/23 Kanagawa governer's cup \#22
\begin{quote}
\crdima{S/N-S}{%
  \begin{minipage}[t]{\br}
     4\s X \\OL:\h 3
  \end{minipage}}%
  {\hand{A42}{J874}{QJ42}{82}}%
  {\hand{J95}{3}{10975}{J9643}}
  {\hand{K}{KQ1062}{AK3}{AQ107}}%
  {\hand{Q108763}{A95}{86}{K5}}%
\end{quote}
\begin{quote}
\begin{bidding}
-  \> -  \> - \> 2\s \\
P \> 3\s \> Dbl \> 4\s \\
P \> P \> Dbl \> a.p.
\end{bidding}
\end{quote}
{\bf Remarks}\\

West leads \h 3. South wins with ace. South leads \s Q which was taken by 
east's \s K. East cashed \h K and \h Q ,while west discards \d 5 and \d 7.
When east cashed \d K, west followed with \d 9. East cashed \c A. As west
shows discarage, east shifted to heart, South discarded \d 8 m loser on loser.
Contract saved a trick, totally three down.

East asked to west why west discared diamonds in this sequence. 
West's diamonds is in odd number of cards.

West answered. My first discard is attitude. My second discard is
present signal. Number of remaining card is three. So lowest.

The present signal was not promised between this pair, veteran players.

We hold the definition of present count.

Present count is a type of standard count signal. A defender's second discard in a suit shows count:\\
A high card shows an even number of remaining cards in the suit.\\
A low card shows an odd number of remaining cards in the suit.
%-------------------88----------------------------------
\vspace{0.5cm}

Next deal shows an example of count signal.\\

\qquad 2004/11/13 Koezuka Cup \#4
\begin{quote}
\crdima{S/vul:Both}{%
  \begin{minipage}[t]{\br}
    3NT \\OL:\s 6
  \end{minipage}}%
  {\hand{J5}{AKQ2}{852}{J643}}%
  {\hand{A9762}{103}{J64}{K102}}
  {\hand{Q843}{9864}{K10}{Q75}}%
  {\hand{K10}{J75}{AQ973}{A98}}%
\end{quote}
\begin{quote}
\begin{bidding}
-  \> -  \> -\> 1\d \\
P \> 1\h \> P \> 1NT \\
P \> 3NT \> a.p.
\end{bidding}
\end{quote}
{\bf Remarks}\\

Opening lead is \s 6 followed by \s J,Q and K. South entered to dummy 
with heart.( East is kind along \h 6 and \h 8.) From dummy a low diamond.
East went up by king as he wanted to lead a spade.
\d A and \d Q won. When west get a lead by \d J ,east discared \s 3.
Can you unlock the key?

In the real game west led \s 2 and defense failed.

There are two reasons to cash \s A.
The one is by present count. If east has even number of cards in remaining,
south's king is alone.

Another authdox counting tells : South has 5 diamonds and 3 hearts and 5 black
cards. If east has originally three cards, he would never throw a spade.

By either methode west can cash \s A. He was crying for a long time.




%-----------------89--------------------
\section{suit preference signal and its application}

The basic suit preference signal is used  at the following
cases below:\\ \index{preference}
when you want your partner let ruff.\\
when you want to tell an entry to you.\\
when you guide your establised suit in NT.\\
These basic signal can be classified into three types:
follow suit. lead,amd discard. In this chapter we talk 
about first two. Discard is in next chapter.\\

Following deal is about follow suit.\\

\qquad 2003/4/26 Japan league \#11
\begin{quote}
\crdima{W/none}{%
  \begin{minipage}[t]{\br}
     6\d \\OL:\s A
  \end{minipage}}%
  {\hand{Q}{K863}{83}{AKQ963}}%
  {\hand{AJ1098764}{Q}{Q6}{J8}}
  {\hand{K52}{A10542}{102}{1053}}%
  {\hand{3}{J97}{AKJ9754}{72}}%
\end{quote}
\begin{quote}
\begin{bidding}
4\s  \> Dbl  \> P \> 5\d \\
a.p.
\end{bidding}
\end{quote}
{\bf Remarks}\\

Opening lead is \s A. East plys \s K ! South is \s 3.
In generally \s K means attitude signal. The location of
\s Q and \s J tells not an attitude, 
The idea for doublton is nonsense. If east is singleton,
south has three spot which contradics slam bidding.
Therefore \s K is a suit preference signal.
Even if west is insensitive , he can understand it.

West leads \h which east wins with ace and heart ruff
continues. Contrace went to down two.

The Interpretation of signal is important. Those of
both players must  be coincident.

Unless rapid defense ,dummy's sold clubs can vanish all loosers.
\vspace{0.5cm}

%-------91--------------------
This deal is played in high level on national pair match\\

\qquad 2006/114 princess takamatu cup \#5
\begin{quote}
\crdima{W/E-W}{%
  \begin{minipage}[t]{\br}
     2\s \\OL:\h K
  \end{minipage}}%
  {\hand{AQ96}{AJ10}{9532}{K6}}%
  {\hand{4}{KQ942}{AjJ6}{J1084}}
  {\hand{K87}{763}{K1084}{Q92}}%
  {\hand{J10532}{85}{Q7}{A753}}%
\end{quote}
\begin{quote}
\begin{bidding}
1\h  \> Dbl  \> 2\h \> 2\s \\
a.p.
\end{bidding}
\end{quote}
{\bf Remarks}\\
%------------------92------------------
\vspace{0.5cm}

\qquad 20049/24 princess takamatu cup \#13
\begin{quote}
\crdima{E/Both}{%
  \begin{minipage}[t]{\br}
     6\s \\OL:\h 4
  \end{minipage}}%
  {\hand{K10643}{3}{10653}{KQ7}}%
  {\hand{9}{AJ1087654}{9872}{-}}
  {\hand{J2}{K92}{KJ4}{J10643}}%
  {\hand{AQ875}{Q}{AQ}{A0852}}%
\end{quote}
\begin{quote}
\begin{bidding}
-  \> -  \> P \> 1\s \\
4\h \> 4\s \> 5\h \> 6\s \\
a.p.
\end{bidding}
\end{quote}
{\bf Remarks}\\

West felt opponent's slam bid was full of confidence.
He hoped a ruff of void suit and led the smallest heart
\h 4 expecting king in partner upon 5H raise.
East was astonished by winning his king.
He returned a club and the went to down one.

The lowest heart means suit-preference. If west wants 
diamond \h 10 is suitable.  The \h J might not be covered
if east has king and queen.


%---------------93---------------
Is suit preference signal a golen rule�E�H

Following heart apperaed in 1NT contract.\\

\qquad \h 1074\\
\h K9862  \qquad \h Aj5\\
\qquad \ \ \h Q3\\
Opening lead is \h 6, fourth best. dummy \h 4.
East wins with ace. Returns \h 5 correctly. South \h Q.
West wins with \h K. 

At this stage west does not know who has \h J. He returns
\h 2 showing 5 cards, East wins with \h J. East led a 
small club 

%---------------94-----------------
\section {Smith signal}
2007/2/19 Yokohama cup \#6
\begin{quote}
\crdima{N/E-W}{%
  \begin{minipage}[t]{\br}
     3NT \\OL:\d 4
  \end{minipage}}%
  {\hand{A1072}{-}{KJ76}{1AJ974}}%
  {\hand{KJ9}{KJ98}{108542}{8}}
  {\hand{Q65}{Q653}{3}{K10653}}%
  {\hand{843}{A10742}{AQ9}{Q2}}%
\end{quote}
\begin{quote}
\begin{bidding}
- \> 1\c  \> P \> 1\h  \\
P \> 1\s \> P \> 2\d \\
P \> 3\d \> P \> 3NT \\
a.p.
\end{bidding}
\end{quote}
In both tables contract was same. Same opening lead.
In one table ,young profeional players suceedwd the defense.
Another table the stupid player failed to defense. The clear
difference is from the signal.





%---------------95------------------
\vspace{0.5cm}
{\bf mission impossible Vienna coup\footnote{
A Vienna Coup is an unblocking play in preparation for a simple squeeze. As the story goes, it was first used in Vienna during a game of Whist.
}
}\\ \index{Vienna coup}

2007/2/10 Shibuya cup \#2
\begin{quote}
\crdima{S/vul:E-W}{%
  \begin{minipage}[t]{\br}
     4\s \\OL:\d A
  \end{minipage}}%
  {\hand{KJ1097}{J2}{Q876}{A3}}%
  {\hand{7}{K863}{A9}{Q108764}}
  {\hand{8642}{9754}{K52}{J5}}%
  {\hand{AQ5}{AQ10}{J1043}{K92}}%
\end{quote}
\begin{quote}
\begin{bidding}
- \> -  \> - \> 1NT \\
P \> 2\h \> P \> 2\s \\
P \> 3NT \> P \> 4\s  \\
a.p.
\end{bidding}
\end{quote}
{\bf Remarks}\\

Opening lead is \d A . West continued \d 9  to east's king.
Third round od diamond is ruffed by west. West leads small club.
Dummu's ace wins. South cashed \s A, West showed out. West had
used his sigleton trump effectively, South cashed \s Q. West sent
an encourage signal with \h 8 and \h 3. South decided not to finess of heart.
In order to squeeze west in heart and club, \h A must be cashed in 
advance, This unblocking play and subsequent squeeze is called
Vienna coup. When dummy took all winners , squeeze succeeded.
This hand is due to good reading of south.
